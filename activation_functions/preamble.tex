% *****************************************************************************
% **************************** Custom Packages ********************************

% *************************  MATHS **************************
\usepackage{amssymb,amsmath,amsthm}
\usepackage{mathtools}
\usepackage{chngcntr}
\usepackage{pdfpages}

\newtheorem{prop}{Proposition}
\newtheorem{definition}{Definition}

\DeclareMathOperator\erfi{erfi}
\DeclareMathOperator\erf{erf}

% ************************* Algorithms and Pseudocode **************************
\usepackage[ruled]{algorithm2e}
%\usepackage{algpseudocode}


% ********************Captions and Hyperreferencing / URL **********************

% Captions: This makes captions of figures use a boldfaced small font.
%\RequirePackage[small,bf]{caption}

\RequirePackage[labelsep=space,tableposition=top]{caption}
\renewcommand{\figurename}{Fig.} %to support older versions of captions.sty


% *************************** Graphics and figures *****************************
\usepackage{subfig}
\usepackage{tikz}
\usepackage{rotating}
\usepackage{wrapfig}
\usepackage{graphicx}
\usepackage{epsfig}
\usepackage{listings}
% Uncomment the following two lines to force Latex to place the figure.
% Use [H] when including graphics. Note 'H' instead of 'h'
\usepackage{float}
\usetikzlibrary{positioning}
\usetikzlibrary{decorations.pathreplacing}
\usetikzlibrary{arrows.meta}
\usetikzlibrary{backgrounds}
\usetikzlibrary{trees,shapes,decorations}
%\restylefloat{figure}

% Subcaption package is also available in the sty folder you can use that by
% uncommenting the following line
% This is for people stuck with older versions of texlive
%\usepackage{sty/caption/subcaption}
%\usepackage{subcaption}

% ********************************** Tables ************************************
\usepackage{xcolor}
\usepackage{booktabs} % For professional looking tables
\usepackage{multirow}
\usepackage{pdflscape}
\usetikzlibrary{decorations.pathreplacing}
\usepackage{pgfplots}
\usepackage{tabularx}
\usepackage[normalem]{ulem}

%\usepackage{multicol}
%\usepackage{longtable}
%\usepackage{tabularx}


% *********************************** SI Units *********************************
\usepackage{siunitx} % use this package module for SI units


% ******************************* Line Spacing *********************************

% Choose linespacing as appropriate. Default is one-half line spacing as per the
% University guidelines

% \doublespacing
% \onehalfspacing
% \singlespacing


% ************************ Formatting / Footnote *******************************

% Don't break enumeration (etc.) across pages in an ugly manner (default 10000)
%\clubpenalty=500
%\widowpenalty=500

%\usepackage[perpage]{footmisc} %Range of footnote options

% ******************************************************************************
% ************************* User Defined Commands ******************************
% ******************************************************************************

% ************************ MATH *****************************************
\newtheorem{theorem}{Theorem}
\newtheorem{lemma}{Lemma}
\usepackage{etoolbox}

% *********** To change the name of Table of Contents / LOF and LOT ************

%\renewcommand{\contentsname}{My Table of Contents}
%\renewcommand{\listfigurename}{My List of Figures}
%\renewcommand{\listtablename}{My List of Tables}

%\counterwithin*{equation}{section}
%\numberwithin{equation}{section}


% ********************** TOC depth and numbering depth *************************

\setcounter{secnumdepth}{3}
\setcounter{tocdepth}{2}



%%%%%%%%%% PLOT %%%%%%%%%

% ******************************* Nomenclature *********************************

% To change the name of the Nomenclature section, uncomment the following line

%\renewcommand{\nomname}{Symbols}


% ********************************* Appendix ***********************************

% The default value of both \appendixtocname and \appendixpagename is `Appendices'. These names can all be changed via:

%\renewcommand{\appendixtocname}{List of appendices}
%\renewcommand{\appendixname}{Appndx}

% *********************** Configure Draft Mode **********************************

% Uncomment to disable figures in `draft'
%\setkeys{Gin}{draft=true}  % set draft to false to enable figures in `draft'

% These options are active only during the draft mode
% Default text is "Draft"
%\SetDraftText{DRAFT}

% Default Watermark location is top. Location (top/bottom)
%\SetDraftWMPosition{bottom}

% Draft Version - default is v1.0
%\SetDraftVersion{v1.1}

% Draft Text grayscale value (should be between 0-black and 1-white)
% Default value is 0.75
%\SetDraftGrayScale{0.8}


% ******************************** Todo Notes **********************************
%% Uncomment the following lines to have todonotes.

%\ifsetDraft
%	\usepackage[colorinlistoftodos]{todonotes}
%	\newcommand{\mynote}[1]{\todo[author=kks32,size=\small,inline,color=green!40]{#1}}
%\else
%	\newcommand{\mynote}[1]{}
%	\newcommand{\listoftodos}{}
%\fi

% Example todo: \mynote{Hey! I have a note}


% -----------------------------------------------------------------
%   Some Colourful Stuff ... 
%   (which I dont need: nvu)
% -----------------------------------------------------------------
\usepackage{hyperref}
\hypersetup{
    colorlinks=true,
    linkcolor=blue,
    filecolor=magenta,
    urlcolor=cyan,
    citecolor=magenta
}

\newcommand{\mynote}[1]{{\color{blue}#1}}

% -----------------------------------------------------------------
%   Defining some useful extensions for MarkDown
% -----------------------------------------------------------------
\usepackage[footnotes,definitionLists,hashEnumerators,smartEllipses, hybrid]{markdown}

\markdownSetup{renderers={
  link = {\href{#2}{#1}}
}}

\hypersetup{%
  colorlinks=true,% hyperlinks will be coloured
  linkcolor=green,% hyperlink text will be green
  linkbordercolor=red,% hyperlink border will be red
}

% -----------------------------------------------------------------
%   To be used as last
% -----------------------------------------------------------------